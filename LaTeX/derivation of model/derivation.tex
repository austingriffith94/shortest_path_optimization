\documentclass{article}
\usepackage{amsmath}
\usepackage{amsfonts}
\usepackage{amssymb}
\usepackage{pdfpages}
\usepackage{authblk}
\newcommand{\pf}[2]{\frac{\partial #1}{\partial #2}}
\newcommand{\ps}[2]{\frac{\partial^2 #1}{\partial #2^2}}

\title{Shortest Path with Variable Edge Failure \\ Financial Optimization}
\author{Austin Griffith}
\affil{Georgia Institute of Technology, Quantitative Finance}
\date{}

\begin{document}
	\maketitle

\section{Problem Setup}
We are given a directed graph $G = (V,E)$. Let $s$ and $t$ both exist on $V$. In this case, $s$ is $0$, and $t$ is the final node value. The nominal travel time on a given edge $(i,j)$ is $c_{ij}$. If the edge fails (gets congested), then the travel time becomes $c_{ij} + d_{ij}$. There are at most $L$ allowable simultaneous edge failures.

The goal is to minimize the worst-case travel cost from $s$ to $t$. Vary $L$ to observe its affect on the model.

\section{Derivation}
The initial setup of the shortest path involves creating a simple minimization mixed integer linear program, using the binary variable $x$.

$$\min c^T x$$
$$
\textrm{s.t.} \quad x \in \{0,1\}
$$
$$
\textrm{for} \quad c_j \in \lbrack\overline{c_{ij}},\overline{c_{ij}} + d_{ij}\rbrack
$$

Since we're minimizing the worst case scenario, we have to also consider the summation of edge failures.

$$
\sum_{j=1}^{n}d_{j}x_{j}
$$

This worst case scenario is determined by the maximum allowable edge failures. Therefore, this summation can be turned into a maximization problem, with decision variable $S$, and variable scalar $L$.

$$
\max_{S \in 1 ... n, |S| = L} \sum_{j \in S}^{}d_{j}x_{j}
$$

From here, this maximization of the worst scenario can now be added to our simple linear program defined at the beginning. This allows us to solve for the optimal worst-case travel scenario.

$$
\underset{x \in \{0,1\}}{\textrm{min}} \lbrack c^T x + \max_{S \in 1 ... n, |S| = L} \sum_{j \in S}^{}d_{j}x_{j} \rbrack
$$

In its current state as a minimzation of a maximization, the program is unsolvable. To deal with this, we first need to solve the inner problem. The initial setup for the inner problem solution is shown below.

$$
\max \sum_{j = 1}^{n} x_{j}d_{j}s_{j}
$$
$$
\textrm{s.t.} \quad \sum_{j =1}^{n} s_{j} = L
$$
$$
s_{j} \in \{0,1\} \quad \forall j
$$

Now that the inner problem has been laid out, we need to solve for the dual. This will set the maximization problem into a minimization, which can then be reinserted into the initial minimization program in a solvable form. In order to solve for the dual, there has to be a relaxation of the binary variable $s$.

$$
\max \sum_{j = 1}^{n} x_{j}d_{j}s_{j}
$$
$$
\textrm{s.t.} \quad \sum_{j =1}^{n} s_{j} = L
$$
$$
0 \leq s_{j} \leq 1 \quad \forall j
$$

Now we can formulate the dual of the relaxation. The dual will be of the following form.

$$ \textrm{objective function} \leq \textrm{relaxation}
$$
$$
\sum_{j}^{} s_{j} \left( \lambda_0 + \lambda_1 \right) \leq L \lambda_0 + \sum_{j}^{} \lambda_j
$$

The constraints of the new dual mixed integer linear program (MILP) are as follows:

$$
\lambda_j \geq 0
$$
$$
\lambda_0 \quad \textrm{u.r.}
$$
$$
\lambda_0 + \lambda_j \geq x_j d_j \quad \forall j
$$
$$
\lambda_j \geq x_j d_j - \lambda_0
$$

To make the combined equation simpler, we can comibine terms. The final two inequalities can be simplified into a pointwise maximum of the two right hand side values.

$$
\lambda_j = max \left(x_j d_j - \lambda_0 ,0\right)
$$


Now that the inner problem has been converted to a minimization problem through duality, we can now add it back into the main MILP.

$$
\underset{x,\lambda_0,z}{\textrm{min}} \overline{c}^T x + L \lambda_0 + \sum_{j = 1}^{n} max \left(x_j d_j - \lambda_0 ,0\right)
$$
$$
x_j \in \{0,1\}
$$
$$
\lambda_0 \quad \textrm{u.r.}
$$

\section{Final Formulation}
Now that the derivation is complete, we can state the final equation for the shortest path model with respect to the problem stated at the beginning of the paper. 

$$
\underset{x,\lambda_0,z}{\textrm{min}} \overline{c}^T x + L \lambda_0 + \sum_{j = 1}^{n} z_{j} x_{j}
$$
$$
\textrm{s.t.} \quad x \in \{0,1\}
$$
$$
\lambda_0 \quad \textrm{u.r.}
$$
$$
z_{j} \geq 0 \quad \forall j
$$
$$
z_{j} \geq d_{j} - \lambda_0 \quad \forall j
$$
$$
\sum_{j}^{} x_{0j} = 1
$$
$$
\sum_{i}^{} x_{i0} = 0
$$
$$
\sum_{i}^{} x_{it} = 1 \quad \textrm{for} \quad t = \textrm{end node}
$$
$$
\sum_{j}^{} x_{tj} = 0 \quad \textrm{for} \quad t = \textrm{end node}
$$
$$
\sum_{i \neq 0, j \neq t}^{} \left( x_{ij} - x_{ji} \right) = 0
$$

In this final formulation, $x$ is a binary decision variable, representing whether a path edge will be taken. $\lambda_0$ is a single continuous, unrestricted decision variable that amplifies the magnitude of the failure value $L$. The final decision variable, $z$, is used to determine which edge will fail, so as to minimize the worst-case scenario. The summations of $x$ values at the end are used to determine the start and end location of the path, and ensure the MILP doesn't end on a node somewhere in the middle.

\end{document}